% !TEX root
% !TEX encoding = UTF-8 Unicode
% -*- coding: UTF-8; -*-
% vim: set fenc=utf-8
\documentclass[french]{beamer}

\mode<presentation> {
\usetheme{Boadilla}
\usecolortheme{seahorse}
\setbeamertemplate{footline}[page number]
\setbeamertemplate{navigation symbols}{}
}

\usepackage{graphicx} % Allows including images
\usepackage{booktabs} % Allows the use of \toprule, \midrule and \bottomrule in tables
\usepackage[french]{babel} % Pour écrire en français
\usepackage[T1]{fontenc}
\usepackage[utf8]{inputenc}

\uselanguage{French}
\languagepath{French}

\title[Lattice Miner]{Conception et implémentation de nouvelles fonctionnalités dans un prototype de fouille de données} % The short title appears at the bottom of every slide, the full title is only on the title page

\author{Kévin Emamirad} % Your name
\institute[UQO] % Your institution as it will appear on the bottom of every slide, may be shorthand to save space
{
Université du Québec en Outaouais \\ % Your institution for the title page
Département d’informatique et d’ingénierie \\
\medskip
\textit{emak01@uqo.ca} % Your email address
}
\date{mercredi 31 mai 2017} % Date, can be changed to a custom date

\begin{document}

\begin{frame}
\titlepage % Print the title page as the first slide
\end{frame}

\begin{frame}
\frametitle{Plan de la présentation} % Table of contents slide, comment this block out to remove it
\tableofcontents % Throughout your presentation, if you choose to use \section{} and \subsection{} commands, these will automatically be printed on this slide as an overview of your presentation
\end{frame}

\section{Introduction}

\begin{frame}
\frametitle{Introduction}
Sed iaculis dapibus gravida. Morbi sed tortor erat, nec interdum arcu. Sed id lorem lectus. Quisque viverra augue id sem ornare non aliquam nibh tristique. Aenean in ligula nisl. Nulla sed tellus ipsum. Donec vestibulum ligula non lorem vulputate fermentum accumsan neque mollis.\\~\\

Sed diam enim, sagittis nec condimentum sit amet, ullamcorper sit amet libero. Aliquam vel dui orci, a porta odio. Nullam id suscipit ipsum. Aenean lobortis commodo sem, ut commodo leo gravida vitae. Pellentesque vehicula ante iaculis arcu pretium rutrum eget sit amet purus. Integer ornare nulla quis neque ultrices lobortis. Vestibulum ultrices tincidunt libero, quis commodo erat ullamcorper id.
\end{frame}

%------------------------------------------------

\begin{frame}
\frametitle{Bullet Points}
\begin{itemize}
\item Lorem ipsum dolor sit amet, consectetur adipiscing elit
\item Aliquam blandit faucibus nisi, sit amet dapibus enim tempus eu
\item Nulla commodo, erat quis gravida posuere, elit lacus lobortis est, quis porttitor odio mauris at libero
\item Nam cursus est eget velit posuere pellentesque
\item Vestibulum faucibus velit a augue condimentum quis convallis nulla gravida
\end{itemize}
\end{frame}

%------------------------------------------------

\begin{frame}
\frametitle{Blocks of Highlighted Text}
\begin{block}{Block 1}
Lorem ipsum dolor sit amet, consectetur adipiscing elit. Integer lectus nisl, ultricies in feugiat rutrum, porttitor sit amet augue. Aliquam ut tortor mauris. Sed volutpat ante purus, quis accumsan dolor.
\end{block}

\begin{block}{Block 2}
Pellentesque sed tellus purus. Class aptent taciti sociosqu ad litora torquent per conubia nostra, per inceptos himenaeos. Vestibulum quis magna at risus dictum tempor eu vitae velit.
\end{block}

\begin{block}{Block 3}
Suspendisse tincidunt sagittis gravida. Curabitur condimentum, enim sed venenatis rutrum, ipsum neque consectetur orci, sed blandit justo nisi ac lacus.
\end{block}
\end{frame}

%------------------------------------------------

\begin{frame}
\frametitle{Multiple Columns}
\begin{columns}[c] % The "c" option specifies centered vertical alignment while the "t" option is used for top vertical alignment

\column{.45\textwidth} % Left column and width
\textbf{Heading}
\begin{enumerate}
\item Statement
\item Explanation
\item Example
\end{enumerate}

\column{.5\textwidth} % Right column and width
Lorem ipsum dolor sit amet, consectetur adipiscing elit. Integer lectus nisl, ultricies in feugiat rutrum, porttitor sit amet augue. Aliquam ut tortor mauris. Sed volutpat ante purus, quis accumsan dolor.

\end{columns}
\end{frame}

%------------------------------------------------
\section{Second Section}
%------------------------------------------------

\begin{frame}
\frametitle{Table}
\begin{table}
\begin{tabular}{l l l}
\toprule
\textbf{Treatments} & \textbf{Response 1} & \textbf{Response 2}\\
\midrule
Treatment 1 & 0.0003262 & 0.562 \\
Treatment 2 & 0.0015681 & 0.910 \\
Treatment 3 & 0.0009271 & 0.296 \\
\bottomrule
\end{tabular}
\caption{Table caption}
\end{table}
\end{frame}

%------------------------------------------------

\begin{frame}
\frametitle{Theorem}
\begin{theorem}[Mass--energy equivalence]
$E = mc^2$
\end{theorem}
\end{frame}

%------------------------------------------------

\begin{frame}[fragile] % Need to use the fragile option when verbatim is used in the slide
\frametitle{Verbatim}
\begin{example}[Theorem Slide Code]
\begin{verbatim}
\begin{frame}
\frametitle{Theorem}
\begin{theorem}[Mass--energy equivalence]
$E = mc^2$
\end{theorem}
\end{frame}\end{verbatim}
\end{example}
\end{frame}

%------------------------------------------------

\begin{frame}
\frametitle{Figure}
Uncomment the code on this slide to include your own image from the same directory as the template .TeX file.
%\begin{figure}
%\includegraphics[width=0.8\linewidth]{test}
%\end{figure}
\end{frame}

%------------------------------------------------

\begin{frame}[fragile] % Need to use the fragile option when verbatim is used in the slide
\frametitle{Citation}
An example of the \verb|\cite| command to cite within the presentation:\\~

This statement requires citation \cite{GW99}.
\end{frame}

%------------------------------------------------

\begin{frame}
\frametitle{References}
\footnotesize{
\begin{thebibliography}{99} % Beamer does not support BibTeX so references must be inserted manually as below
\bibitem{GW99}
{\sc Ganter, B. et Wille, R.}
\newblock {\em Formal Concept Analysis: Mathematical Foundations}.
\newblock Springer-Verlag New York, Inc., 1999.

\bibitem{3}
{\sc Guigues J.~L. et Duquenne~V.}
\newblock Familles minimales d'implications informatives r{\'e}sultant d'un
  tableau de donn{\'e}es binaires.
\newblock {\em Math\'ematiques et Sciences Humaines 95\/} (1986), 5--18.

\bibitem{Ganter04}
{\sc Ganter, B. et Obiedkov, S.~A.}
\newblock Implications in triadic formal contexts.
\newblock In {\em ICCS\/} (2004), pp.~186--195.

\bibitem{Roberge}
{\sc Roberge, G.}
\newblock Visualisation des r{\'e}sultats d'une fouille de donn{\'e}es dans les
  treillis de concepts.
\newblock M{\'e}moire de ma\^{i}trise, Universit{\'e} du Qu{\'e}bec en Outaouais, 2007.

\bibitem{Missaoui2012}
{\sc Missaoui, R., Nourine, L. et Renaud, Y.}
\newblock Computing implications with negation from a formal context.
\newblock {\em Fundam. Inf. 115}, 4 (Dec. 2012), 357--375.

\end{thebibliography}
}
\end{frame}

\begin{frame}
\frametitle{References}
\footnotesize{
\begin{thebibliography}{99} % Beamer does not support BibTeX so references must be inserted manually as below

\bibitem{Missaoui2017}
{\sc Missaoui, R., et Emamirad, K.}
\newblock Lattice Miner 2.0: A Formal Concept Analysis Tool.
\newblock In {\em Supplementary Proc. of ICFCA, Rennes, France, 2017\/} (2017),
  pp.~91--94.
\end{thebibliography}
}
\end{frame}
%------------------------------------------------

\begin{frame}
\Huge{\centerline{Questions ?}}
\end{frame}

%----------------------------------------------------------------------------------------

\end{document} 